% Options for packages loaded elsewhere
\PassOptionsToPackage{unicode}{hyperref}
\PassOptionsToPackage{hyphens}{url}
%
\documentclass[
]{article}
\usepackage{amsmath,amssymb}
\usepackage{lmodern}
\usepackage{ifxetex,ifluatex}
\ifnum 0\ifxetex 1\fi\ifluatex 1\fi=0 % if pdftex
  \usepackage[T1]{fontenc}
  \usepackage[utf8]{inputenc}
  \usepackage{textcomp} % provide euro and other symbols
\else % if luatex or xetex
  \usepackage{unicode-math}
  \defaultfontfeatures{Scale=MatchLowercase}
  \defaultfontfeatures[\rmfamily]{Ligatures=TeX,Scale=1}
\fi
% Use upquote if available, for straight quotes in verbatim environments
\IfFileExists{upquote.sty}{\usepackage{upquote}}{}
\IfFileExists{microtype.sty}{% use microtype if available
  \usepackage[]{microtype}
  \UseMicrotypeSet[protrusion]{basicmath} % disable protrusion for tt fonts
}{}
\makeatletter
\@ifundefined{KOMAClassName}{% if non-KOMA class
  \IfFileExists{parskip.sty}{%
    \usepackage{parskip}
  }{% else
    \setlength{\parindent}{0pt}
    \setlength{\parskip}{6pt plus 2pt minus 1pt}}
}{% if KOMA class
  \KOMAoptions{parskip=half}}
\makeatother
\usepackage{xcolor}
\IfFileExists{xurl.sty}{\usepackage{xurl}}{} % add URL line breaks if available
\IfFileExists{bookmark.sty}{\usepackage{bookmark}}{\usepackage{hyperref}}
\hypersetup{
  pdftitle={Introduction to R},
  hidelinks,
  pdfcreator={LaTeX via pandoc}}
\urlstyle{same} % disable monospaced font for URLs
\usepackage[margin=1in]{geometry}
\usepackage{color}
\usepackage{fancyvrb}
\newcommand{\VerbBar}{|}
\newcommand{\VERB}{\Verb[commandchars=\\\{\}]}
\DefineVerbatimEnvironment{Highlighting}{Verbatim}{commandchars=\\\{\}}
% Add ',fontsize=\small' for more characters per line
\usepackage{framed}
\definecolor{shadecolor}{RGB}{248,248,248}
\newenvironment{Shaded}{\begin{snugshade}}{\end{snugshade}}
\newcommand{\AlertTok}[1]{\textcolor[rgb]{0.94,0.16,0.16}{#1}}
\newcommand{\AnnotationTok}[1]{\textcolor[rgb]{0.56,0.35,0.01}{\textbf{\textit{#1}}}}
\newcommand{\AttributeTok}[1]{\textcolor[rgb]{0.77,0.63,0.00}{#1}}
\newcommand{\BaseNTok}[1]{\textcolor[rgb]{0.00,0.00,0.81}{#1}}
\newcommand{\BuiltInTok}[1]{#1}
\newcommand{\CharTok}[1]{\textcolor[rgb]{0.31,0.60,0.02}{#1}}
\newcommand{\CommentTok}[1]{\textcolor[rgb]{0.56,0.35,0.01}{\textit{#1}}}
\newcommand{\CommentVarTok}[1]{\textcolor[rgb]{0.56,0.35,0.01}{\textbf{\textit{#1}}}}
\newcommand{\ConstantTok}[1]{\textcolor[rgb]{0.00,0.00,0.00}{#1}}
\newcommand{\ControlFlowTok}[1]{\textcolor[rgb]{0.13,0.29,0.53}{\textbf{#1}}}
\newcommand{\DataTypeTok}[1]{\textcolor[rgb]{0.13,0.29,0.53}{#1}}
\newcommand{\DecValTok}[1]{\textcolor[rgb]{0.00,0.00,0.81}{#1}}
\newcommand{\DocumentationTok}[1]{\textcolor[rgb]{0.56,0.35,0.01}{\textbf{\textit{#1}}}}
\newcommand{\ErrorTok}[1]{\textcolor[rgb]{0.64,0.00,0.00}{\textbf{#1}}}
\newcommand{\ExtensionTok}[1]{#1}
\newcommand{\FloatTok}[1]{\textcolor[rgb]{0.00,0.00,0.81}{#1}}
\newcommand{\FunctionTok}[1]{\textcolor[rgb]{0.00,0.00,0.00}{#1}}
\newcommand{\ImportTok}[1]{#1}
\newcommand{\InformationTok}[1]{\textcolor[rgb]{0.56,0.35,0.01}{\textbf{\textit{#1}}}}
\newcommand{\KeywordTok}[1]{\textcolor[rgb]{0.13,0.29,0.53}{\textbf{#1}}}
\newcommand{\NormalTok}[1]{#1}
\newcommand{\OperatorTok}[1]{\textcolor[rgb]{0.81,0.36,0.00}{\textbf{#1}}}
\newcommand{\OtherTok}[1]{\textcolor[rgb]{0.56,0.35,0.01}{#1}}
\newcommand{\PreprocessorTok}[1]{\textcolor[rgb]{0.56,0.35,0.01}{\textit{#1}}}
\newcommand{\RegionMarkerTok}[1]{#1}
\newcommand{\SpecialCharTok}[1]{\textcolor[rgb]{0.00,0.00,0.00}{#1}}
\newcommand{\SpecialStringTok}[1]{\textcolor[rgb]{0.31,0.60,0.02}{#1}}
\newcommand{\StringTok}[1]{\textcolor[rgb]{0.31,0.60,0.02}{#1}}
\newcommand{\VariableTok}[1]{\textcolor[rgb]{0.00,0.00,0.00}{#1}}
\newcommand{\VerbatimStringTok}[1]{\textcolor[rgb]{0.31,0.60,0.02}{#1}}
\newcommand{\WarningTok}[1]{\textcolor[rgb]{0.56,0.35,0.01}{\textbf{\textit{#1}}}}
\usepackage{graphicx}
\makeatletter
\def\maxwidth{\ifdim\Gin@nat@width>\linewidth\linewidth\else\Gin@nat@width\fi}
\def\maxheight{\ifdim\Gin@nat@height>\textheight\textheight\else\Gin@nat@height\fi}
\makeatother
% Scale images if necessary, so that they will not overflow the page
% margins by default, and it is still possible to overwrite the defaults
% using explicit options in \includegraphics[width, height, ...]{}
\setkeys{Gin}{width=\maxwidth,height=\maxheight,keepaspectratio}
% Set default figure placement to htbp
\makeatletter
\def\fps@figure{htbp}
\makeatother
\setlength{\emergencystretch}{3em} % prevent overfull lines
\providecommand{\tightlist}{%
  \setlength{\itemsep}{0pt}\setlength{\parskip}{0pt}}
\setcounter{secnumdepth}{-\maxdimen} % remove section numbering
\ifluatex
  \usepackage{selnolig}  % disable illegal ligatures
\fi

\title{Introduction to R}
\author{}
\date{\vspace{-2.5em}}

\begin{document}
\maketitle

\hypertarget{r-markdown}{%
\subsection{R Markdown}\label{r-markdown}}

R is an object-oriented language: everything in R is an object.

\hypertarget{r-has-6-basic-data-types}{%
\subparagraph{R has 6 basic data
types:}\label{r-has-6-basic-data-types}}

\begin{itemize}
\item
  character

  : ``a'', ``ciao''
\item
  numeric

  : 2, 15.5
\item
  integer

  : 2L (the L tells R to store this as an integer)
\item
  logical

  : TRUE, FALSE (aka boolean)
\item
  complex

  : 1+4i (complex numbers with real and imaginary parts)
\end{itemize}

\emph{Atomic} means that the vector only holds data of a single data
type.

\hypertarget{examining-objects}{%
\subparagraph{Examining objects}\label{examining-objects}}

R provides many functions to examine features of vectors and objects in
general, eg:

\begin{itemize}
\tightlist
\item
  class() - what kind of object is it (high-level)?
\item
  typeof() - what is the object's data type (low-level)?
\item
  length() - how long is it? What about two dimensional objects?
\item
  attributes() - does it have any metadata?
\end{itemize}

Example

\begin{Shaded}
\begin{Highlighting}[]
\NormalTok{x }\OtherTok{\textless{}{-}} \StringTok{"Hola"}
\FunctionTok{typeof}\NormalTok{(x)}
\end{Highlighting}
\end{Shaded}

\begin{verbatim}
## [1] "character"
\end{verbatim}

\begin{Shaded}
\begin{Highlighting}[]
\FunctionTok{attributes}\NormalTok{(x)}
\end{Highlighting}
\end{Shaded}

\begin{verbatim}
## NULL
\end{verbatim}

\begin{Shaded}
\begin{Highlighting}[]
\NormalTok{s }\OtherTok{\textless{}{-}} \DecValTok{1}\SpecialCharTok{:}\DecValTok{5}
\NormalTok{s}
\end{Highlighting}
\end{Shaded}

\begin{verbatim}
## [1] 1 2 3 4 5
\end{verbatim}

\begin{Shaded}
\begin{Highlighting}[]
\FunctionTok{length}\NormalTok{(s)}
\end{Highlighting}
\end{Shaded}

\begin{verbatim}
## [1] 5
\end{verbatim}

\begin{Shaded}
\begin{Highlighting}[]
\FunctionTok{typeof}\NormalTok{(s)}
\end{Highlighting}
\end{Shaded}

\begin{verbatim}
## [1] "integer"
\end{verbatim}

\begin{Shaded}
\begin{Highlighting}[]
\CommentTok{\#convert to numeric}
\NormalTok{num\_s }\OtherTok{\textless{}{-}} \FunctionTok{as.numeric}\NormalTok{(s)}
\NormalTok{num\_s}
\end{Highlighting}
\end{Shaded}

\begin{verbatim}
## [1] 1 2 3 4 5
\end{verbatim}

\begin{Shaded}
\begin{Highlighting}[]
\FunctionTok{typeof}\NormalTok{(num\_s)}
\end{Highlighting}
\end{Shaded}

\begin{verbatim}
## [1] "double"
\end{verbatim}

\hypertarget{data-structures}{%
\subsection{Data structures}\label{data-structures}}

R has many data structures. The main ones are the following:

\begin{itemize}
\tightlist
\item
  atomic vector
\item
  list
\item
  matrix
\item
  data frame
\item
  factors
\item
  time series
\end{itemize}

\hypertarget{vectors}{%
\subsubsection{Vectors}\label{vectors}}

The most common and basic data structure in R. Vectors can be one of two
types:

\begin{itemize}
\tightlist
\item
  atomic vectors\\
\item
  lists
\end{itemize}

In general the term \emph{vector} is used in reference to the atomic
types only.

A vector is a collection of elements usually of mode: character,
logical, integer or numeric. it is a one-dimensional data structure and
it is homogeneous, i.e.~each element is of the same type.

\hypertarget{create-a-vector}{%
\subparagraph{Create a vector}\label{create-a-vector}}

To create an empty vector use vector(). The default mode is logical. You
can be more explicit (as shown in the examples below.) It is more common
to use direct constructors such as character(), numeric(), etc.

\begin{Shaded}
\begin{Highlighting}[]
\FunctionTok{vector}\NormalTok{()}
\end{Highlighting}
\end{Shaded}

\begin{verbatim}
## logical(0)
\end{verbatim}

\begin{Shaded}
\begin{Highlighting}[]
\CommentTok{\# a vector of mode \textquotesingle{}character\textquotesingle{} composed of 5 elements}
\FunctionTok{vector}\NormalTok{(}\StringTok{"character"}\NormalTok{, }\AttributeTok{length =} \DecValTok{5}\NormalTok{) }
\end{Highlighting}
\end{Shaded}

\begin{verbatim}
## [1] "" "" "" "" ""
\end{verbatim}

\begin{Shaded}
\begin{Highlighting}[]
\CommentTok{\# same same, but using only the constructor directly.}
\FunctionTok{character}\NormalTok{(}\DecValTok{5}\NormalTok{) }\CommentTok{\# a \textquotesingle{}character\textquotesingle{} vector composed of 5 elements}
\end{Highlighting}
\end{Shaded}

\begin{verbatim}
## [1] "" "" "" "" ""
\end{verbatim}

\begin{Shaded}
\begin{Highlighting}[]
\FunctionTok{numeric}\NormalTok{(}\DecValTok{5}\NormalTok{)   }\CommentTok{\# a \textquotesingle{}numeric\textquotesingle{} vector with 5 elements}
\end{Highlighting}
\end{Shaded}

\begin{verbatim}
## [1] 0 0 0 0 0
\end{verbatim}

One can create a vector directly specifying their content. R will guess
the appropriate mode of storage.

\begin{Shaded}
\begin{Highlighting}[]
\CommentTok{\# this will be treated as double precision real numbers}
\NormalTok{v }\OtherTok{\textless{}{-}} \FunctionTok{c}\NormalTok{(}\DecValTok{1}\NormalTok{, }\DecValTok{2}\NormalTok{, }\DecValTok{3}\NormalTok{)}
\CommentTok{\# to force treating as integer numbers}
\NormalTok{v\_num }\OtherTok{\textless{}{-}} \FunctionTok{c}\NormalTok{(1L, 2L, 3L)}
\CommentTok{\# to force treating as integer numbers it can be casted as well}
\NormalTok{v }\OtherTok{\textless{}{-}} \FunctionTok{as.integer}\NormalTok{(}\FunctionTok{c}\NormalTok{(}\DecValTok{1}\NormalTok{, }\DecValTok{2}\NormalTok{, }\DecValTok{3}\NormalTok{))}

\CommentTok{\#create a vector of mode logical}
\NormalTok{y }\OtherTok{\textless{}{-}} \FunctionTok{c}\NormalTok{(}\ConstantTok{TRUE}\NormalTok{, }\ConstantTok{TRUE}\NormalTok{, }\ConstantTok{FALSE}\NormalTok{, }\ConstantTok{FALSE}\NormalTok{)}

\CommentTok{\#create a vectore of mode character}
\NormalTok{z }\OtherTok{\textless{}{-}} \FunctionTok{c}\NormalTok{(}\StringTok{"Fabi"}\NormalTok{, }\StringTok{"Oriol"}\NormalTok{, }\StringTok{"Alessandro"}\NormalTok{, }\StringTok{"Jasper"}\NormalTok{, }\StringTok{"Alex"}\NormalTok{, }\StringTok{"David"}\NormalTok{, }\StringTok{"Alejandro"}\NormalTok{, }\StringTok{"Laura"}\NormalTok{, }\StringTok{"Jasper"}\NormalTok{, }\StringTok{"Kevin"}\NormalTok{)}
\end{Highlighting}
\end{Shaded}

\hypertarget{examining-vectors}{%
\subparagraph{Examining vectors}\label{examining-vectors}}

\begin{Shaded}
\begin{Highlighting}[]
\FunctionTok{typeof}\NormalTok{(v)}
\end{Highlighting}
\end{Shaded}

\begin{verbatim}
## [1] "integer"
\end{verbatim}

\begin{Shaded}
\begin{Highlighting}[]
\FunctionTok{length}\NormalTok{(v)}
\end{Highlighting}
\end{Shaded}

\begin{verbatim}
## [1] 3
\end{verbatim}

\begin{Shaded}
\begin{Highlighting}[]
\FunctionTok{class}\NormalTok{(v)}
\end{Highlighting}
\end{Shaded}

\begin{verbatim}
## [1] "integer"
\end{verbatim}

\begin{Shaded}
\begin{Highlighting}[]
\FunctionTok{str}\NormalTok{(v)}
\end{Highlighting}
\end{Shaded}

\begin{verbatim}
##  int [1:3] 1 2 3
\end{verbatim}

\begin{Shaded}
\begin{Highlighting}[]
\FunctionTok{typeof}\NormalTok{(z)}
\end{Highlighting}
\end{Shaded}

\begin{verbatim}
## [1] "character"
\end{verbatim}

\begin{Shaded}
\begin{Highlighting}[]
\FunctionTok{length}\NormalTok{(z)}
\end{Highlighting}
\end{Shaded}

\begin{verbatim}
## [1] 10
\end{verbatim}

\begin{Shaded}
\begin{Highlighting}[]
\FunctionTok{class}\NormalTok{(z)}
\end{Highlighting}
\end{Shaded}

\begin{verbatim}
## [1] "character"
\end{verbatim}

\begin{Shaded}
\begin{Highlighting}[]
\FunctionTok{str}\NormalTok{(z)}
\end{Highlighting}
\end{Shaded}

\begin{verbatim}
##  chr [1:10] "Fabi" "Oriol" "Alessandro" "Jasper" "Alex" "David" "Alejandro" ...
\end{verbatim}

\hypertarget{adding-elements-to-a-vector}{%
\subparagraph{Adding elements to a
vector}\label{adding-elements-to-a-vector}}

\begin{Shaded}
\begin{Highlighting}[]
\CommentTok{\#add elements one by one}
\NormalTok{en\_z }\OtherTok{\textless{}{-}} \FunctionTok{c}\NormalTok{(z, }\StringTok{"Kevin"}\NormalTok{)}
\NormalTok{en\_z }\OtherTok{\textless{}{-}} \FunctionTok{c}\NormalTok{(z, }\StringTok{"Ariadna"}\NormalTok{)}
\NormalTok{en\_z}
\end{Highlighting}
\end{Shaded}

\begin{verbatim}
##  [1] "Fabi"       "Oriol"      "Alessandro" "Jasper"     "Alex"      
##  [6] "David"      "Alejandro"  "Laura"      "Jasper"     "Kevin"     
## [11] "Ariadna"
\end{verbatim}

\begin{Shaded}
\begin{Highlighting}[]
\CommentTok{\#add two elements at the same time basically concatenating vectors}
\NormalTok{En\_z }\OtherTok{\textless{}{-}} \FunctionTok{c}\NormalTok{(z, }\FunctionTok{c}\NormalTok{(}\StringTok{"Kevin"}\NormalTok{, }\StringTok{"Ariadna"}\NormalTok{))}
\NormalTok{En\_z}
\end{Highlighting}
\end{Shaded}

\begin{verbatim}
##  [1] "Fabi"       "Oriol"      "Alessandro" "Jasper"     "Alex"      
##  [6] "David"      "Alejandro"  "Laura"      "Jasper"     "Kevin"     
## [11] "Kevin"      "Ariadna"
\end{verbatim}

\hypertarget{create-vectors-from-sequences}{%
\subparagraph{Create Vectors from
Sequences}\label{create-vectors-from-sequences}}

\begin{Shaded}
\begin{Highlighting}[]
\CommentTok{\#sequence of numbers}
\NormalTok{int\_series }\OtherTok{\textless{}{-}} \DecValTok{1}\SpecialCharTok{:}\DecValTok{5}
\NormalTok{int\_series}
\end{Highlighting}
\end{Shaded}

\begin{verbatim}
## [1] 1 2 3 4 5
\end{verbatim}

\begin{Shaded}
\begin{Highlighting}[]
\NormalTok{series }\OtherTok{\textless{}{-}} \FloatTok{1.5}\SpecialCharTok{:}\FloatTok{5.5}
\NormalTok{series}
\end{Highlighting}
\end{Shaded}

\begin{verbatim}
## [1] 1.5 2.5 3.5 4.5 5.5
\end{verbatim}

\begin{Shaded}
\begin{Highlighting}[]
\CommentTok{\#same same}
\FunctionTok{seq}\NormalTok{(}\DecValTok{10}\NormalTok{)}
\end{Highlighting}
\end{Shaded}

\begin{verbatim}
##  [1]  1  2  3  4  5  6  7  8  9 10
\end{verbatim}

\begin{Shaded}
\begin{Highlighting}[]
\FunctionTok{seq}\NormalTok{(}\AttributeTok{from =} \DecValTok{1}\NormalTok{, }\AttributeTok{to =} \DecValTok{5}\NormalTok{, }\AttributeTok{by =} \FloatTok{0.5}\NormalTok{)}
\end{Highlighting}
\end{Shaded}

\begin{verbatim}
## [1] 1.0 1.5 2.0 2.5 3.0 3.5 4.0 4.5 5.0
\end{verbatim}

\hypertarget{selecting-elements}{%
\subparagraph{Selecting elements}\label{selecting-elements}}

\begin{Shaded}
\begin{Highlighting}[]
\CommentTok{\#sequence of numbers}
\NormalTok{int\_series }\OtherTok{\textless{}{-}} \DecValTok{1}\SpecialCharTok{:}\DecValTok{5}
\NormalTok{int\_series[}\FunctionTok{c}\NormalTok{(}\DecValTok{1}\SpecialCharTok{:}\DecValTok{2}\NormalTok{,}\DecValTok{5}\NormalTok{)]}
\end{Highlighting}
\end{Shaded}

\begin{verbatim}
## [1] 1 2 5
\end{verbatim}

\hypertarget{special-values}{%
\subparagraph{Special values}\label{special-values}}

\textbf{NA} = Not Available

\begin{Shaded}
\begin{Highlighting}[]
\NormalTok{x }\OtherTok{\textless{}{-}} \FunctionTok{c}\NormalTok{(}\FloatTok{0.5}\NormalTok{, }\ConstantTok{NA}\NormalTok{, }\FloatTok{0.7}\NormalTok{)}
\CommentTok{\#check if there are NA elements one by one}
\FunctionTok{is.na}\NormalTok{(x)}
\end{Highlighting}
\end{Shaded}

\begin{verbatim}
## [1] FALSE  TRUE FALSE
\end{verbatim}

\begin{Shaded}
\begin{Highlighting}[]
\CommentTok{\#check if there is any NA element}
\FunctionTok{anyNA}\NormalTok{(x)}
\end{Highlighting}
\end{Shaded}

\begin{verbatim}
## [1] TRUE
\end{verbatim}

\textbf{Inf} = infinite

\begin{Shaded}
\begin{Highlighting}[]
\DecValTok{4}\SpecialCharTok{/}\DecValTok{0}
\end{Highlighting}
\end{Shaded}

\begin{verbatim}
## [1] Inf
\end{verbatim}

\textbf{NaN} = Not a Number = undefined value.

\begin{Shaded}
\begin{Highlighting}[]
\DecValTok{0}\SpecialCharTok{/}\DecValTok{0}
\end{Highlighting}
\end{Shaded}

\begin{verbatim}
## [1] NaN
\end{verbatim}

\begin{Shaded}
\begin{Highlighting}[]
\ConstantTok{Inf}\SpecialCharTok{/}\ConstantTok{Inf}
\end{Highlighting}
\end{Shaded}

\begin{verbatim}
## [1] NaN
\end{verbatim}

\begin{Shaded}
\begin{Highlighting}[]
\ConstantTok{Inf}\SpecialCharTok{/}\DecValTok{0}
\end{Highlighting}
\end{Shaded}

\begin{verbatim}
## [1] Inf
\end{verbatim}

\hypertarget{vectors-with-mixed-types-elements}{%
\subparagraph{Vectors with mixed types
elements}\label{vectors-with-mixed-types-elements}}

\begin{Shaded}
\begin{Highlighting}[]
\CommentTok{\#r automatically converts to a single type}
\NormalTok{m1 }\OtherTok{\textless{}{-}} \FunctionTok{c}\NormalTok{(}\FloatTok{1.7}\NormalTok{, }\StringTok{"a"}\NormalTok{)}
\NormalTok{m1}
\end{Highlighting}
\end{Shaded}

\begin{verbatim}
## [1] "1.7" "a"
\end{verbatim}

\begin{Shaded}
\begin{Highlighting}[]
\NormalTok{m2 }\OtherTok{\textless{}{-}} \FunctionTok{c}\NormalTok{(}\ConstantTok{TRUE}\NormalTok{, }\DecValTok{2}\NormalTok{)}
\NormalTok{m2}
\end{Highlighting}
\end{Shaded}

\begin{verbatim}
## [1] 1 2
\end{verbatim}

\begin{Shaded}
\begin{Highlighting}[]
\NormalTok{m3 }\OtherTok{\textless{}{-}} \FunctionTok{c}\NormalTok{(}\ConstantTok{FALSE}\NormalTok{, }\DecValTok{3}\NormalTok{)}
\NormalTok{m3}
\end{Highlighting}
\end{Shaded}

\begin{verbatim}
## [1] 0 3
\end{verbatim}

\begin{Shaded}
\begin{Highlighting}[]
\NormalTok{m4 }\OtherTok{\textless{}{-}} \FunctionTok{c}\NormalTok{(}\StringTok{"a"}\NormalTok{, }\ConstantTok{TRUE}\NormalTok{)}
\NormalTok{m4}
\end{Highlighting}
\end{Shaded}

\begin{verbatim}
## [1] "a"    "TRUE"
\end{verbatim}

\hypertarget{matrices}{%
\subsubsection{Matrices}\label{matrices}}

Multi dimensional atomic vectors, as such they are homogeneous
structures containing just one data type. Generally 2 dimensions: rows
and columns.

\hypertarget{create-a-matrix}{%
\subparagraph{Create a matrix}\label{create-a-matrix}}

\begin{Shaded}
\begin{Highlighting}[]
\NormalTok{m }\OtherTok{\textless{}{-}} \FunctionTok{matrix}\NormalTok{(}\AttributeTok{nrow =} \DecValTok{3}\NormalTok{, }\AttributeTok{ncol =} \DecValTok{4}\NormalTok{)}
\NormalTok{m}
\end{Highlighting}
\end{Shaded}

\begin{verbatim}
##      [,1] [,2] [,3] [,4]
## [1,]   NA   NA   NA   NA
## [2,]   NA   NA   NA   NA
## [3,]   NA   NA   NA   NA
\end{verbatim}

\begin{Shaded}
\begin{Highlighting}[]
\CommentTok{\#print dimensions nrows and ncols}
\FunctionTok{dim}\NormalTok{(m)}
\end{Highlighting}
\end{Shaded}

\begin{verbatim}
## [1] 3 4
\end{verbatim}

\begin{Shaded}
\begin{Highlighting}[]
\CommentTok{\#high level type}
\FunctionTok{class}\NormalTok{(m)}
\end{Highlighting}
\end{Shaded}

\begin{verbatim}
## [1] "matrix"
\end{verbatim}

\begin{Shaded}
\begin{Highlighting}[]
\CommentTok{\#low level type}
\FunctionTok{typeof}\NormalTok{(m)}
\end{Highlighting}
\end{Shaded}

\begin{verbatim}
## [1] "logical"
\end{verbatim}

\begin{Shaded}
\begin{Highlighting}[]
\NormalTok{m }\OtherTok{\textless{}{-}} \FunctionTok{matrix}\NormalTok{(}\FunctionTok{c}\NormalTok{(}\DecValTok{1}\NormalTok{,}\DecValTok{3}\NormalTok{))}
\NormalTok{m}
\end{Highlighting}
\end{Shaded}

\begin{verbatim}
##      [,1]
## [1,]    1
## [2,]    3
\end{verbatim}

\begin{Shaded}
\begin{Highlighting}[]
\CommentTok{\#print dimensions nrows and ncols}
\FunctionTok{dim}\NormalTok{(m)}
\end{Highlighting}
\end{Shaded}

\begin{verbatim}
## [1] 2 1
\end{verbatim}

\begin{Shaded}
\begin{Highlighting}[]
\CommentTok{\#high level type}
\FunctionTok{class}\NormalTok{(m)}
\end{Highlighting}
\end{Shaded}

\begin{verbatim}
## [1] "matrix"
\end{verbatim}

\begin{Shaded}
\begin{Highlighting}[]
\CommentTok{\#low level type}
\FunctionTok{typeof}\NormalTok{(m)}
\end{Highlighting}
\end{Shaded}

\begin{verbatim}
## [1] "double"
\end{verbatim}

\begin{Shaded}
\begin{Highlighting}[]
\NormalTok{mq }\OtherTok{\textless{}{-}} \FunctionTok{matrix}\NormalTok{(}
  \FunctionTok{c}\NormalTok{(}\DecValTok{1}\NormalTok{, }\DecValTok{1}\NormalTok{, }\DecValTok{2}\NormalTok{, }\DecValTok{3}\NormalTok{),}
  \AttributeTok{nrow =} \DecValTok{2}\NormalTok{,}
  \AttributeTok{ncol =} \DecValTok{6}\NormalTok{)}
\NormalTok{mq}
\end{Highlighting}
\end{Shaded}

\begin{verbatim}
##      [,1] [,2] [,3] [,4] [,5] [,6]
## [1,]    1    2    1    2    1    2
## [2,]    1    3    1    3    1    3
\end{verbatim}

Matrices are filled columns wise.

\begin{Shaded}
\begin{Highlighting}[]
\NormalTok{m }\OtherTok{\textless{}{-}} \FunctionTok{matrix}\NormalTok{(}\DecValTok{1}\SpecialCharTok{:}\DecValTok{6}\NormalTok{, }\AttributeTok{nrow =} \DecValTok{2}\NormalTok{, }\AttributeTok{ncol =} \DecValTok{3}\NormalTok{)}
\NormalTok{m}
\end{Highlighting}
\end{Shaded}

\begin{verbatim}
##      [,1] [,2] [,3]
## [1,]    1    3    5
## [2,]    2    4    6
\end{verbatim}

\hypertarget{select-matrix-elements}{%
\subparagraph{Select matrix elements}\label{select-matrix-elements}}

To reference a matrix element: specify the index along each dimension in
single square brackets, e.g.~m{[}row\_index, column\_index{]}.

\begin{Shaded}
\begin{Highlighting}[]
\NormalTok{mq[}\DecValTok{1}\NormalTok{,}\DecValTok{4}\NormalTok{]}
\end{Highlighting}
\end{Shaded}

\begin{verbatim}
## [1] 2
\end{verbatim}

\begin{Shaded}
\begin{Highlighting}[]
\NormalTok{mq[}\DecValTok{2}\NormalTok{,}\DecValTok{4}\NormalTok{]}
\end{Highlighting}
\end{Shaded}

\begin{verbatim}
## [1] 3
\end{verbatim}

\begin{Shaded}
\begin{Highlighting}[]
\FunctionTok{typeof}\NormalTok{(mq[}\DecValTok{1}\NormalTok{,}\DecValTok{1}\NormalTok{]) }\SpecialCharTok{==} \FunctionTok{typeof}\NormalTok{(mq)}
\end{Highlighting}
\end{Shaded}

\begin{verbatim}
## [1] TRUE
\end{verbatim}

\begin{Shaded}
\begin{Highlighting}[]
\NormalTok{mq[}\DecValTok{9}\NormalTok{] }\CommentTok{\#this selects the ninth element column{-}wise}
\end{Highlighting}
\end{Shaded}

\begin{verbatim}
## [1] 1
\end{verbatim}

\hypertarget{create-a-matrix-from-a-vector}{%
\subparagraph{Create a matrix from a
vector}\label{create-a-matrix-from-a-vector}}

\begin{Shaded}
\begin{Highlighting}[]
\NormalTok{v }\OtherTok{\textless{}{-}} \DecValTok{1}\SpecialCharTok{:}\DecValTok{10}
\CommentTok{\#assigning dimension to the matrix}
\FunctionTok{dim}\NormalTok{(v) }\OtherTok{\textless{}{-}} \FunctionTok{c}\NormalTok{(}\DecValTok{2}\NormalTok{, }\DecValTok{5}\NormalTok{) }
\CommentTok{\#dim(v) \textless{}{-} c(5, 2) }
\NormalTok{v}
\end{Highlighting}
\end{Shaded}

\begin{verbatim}
##      [,1] [,2] [,3] [,4] [,5]
## [1,]    1    3    5    7    9
## [2,]    2    4    6    8   10
\end{verbatim}

\hypertarget{create-a-matrix-binding-rows-or-columns}{%
\subparagraph{Create a matrix binding rows or
columns}\label{create-a-matrix-binding-rows-or-columns}}

\begin{itemize}
\tightlist
\item
  rbind() to bind rows\\
\item
  cbind() to bind columns
\end{itemize}

NB Dimesions must match!

\begin{Shaded}
\begin{Highlighting}[]
\NormalTok{c1 }\OtherTok{\textless{}{-}} \DecValTok{3}\SpecialCharTok{:}\DecValTok{6}
\NormalTok{c2 }\OtherTok{\textless{}{-}} \DecValTok{5}\SpecialCharTok{:}\DecValTok{8}
\FunctionTok{cbind}\NormalTok{(c1, c2)}
\end{Highlighting}
\end{Shaded}

\begin{verbatim}
##      c1 c2
## [1,]  3  5
## [2,]  4  6
## [3,]  5  7
## [4,]  6  8
\end{verbatim}

\begin{Shaded}
\begin{Highlighting}[]
\NormalTok{r1 }\OtherTok{\textless{}{-}} \DecValTok{3}\SpecialCharTok{:}\DecValTok{6}
\NormalTok{r2 }\OtherTok{\textless{}{-}} \DecValTok{5}\SpecialCharTok{:}\DecValTok{8}
\NormalTok{r3 }\OtherTok{\textless{}{-}} \DecValTok{9}\SpecialCharTok{:}\DecValTok{12}
\FunctionTok{rbind}\NormalTok{(c1, c2, r3)}
\end{Highlighting}
\end{Shaded}

\begin{verbatim}
##    [,1] [,2] [,3] [,4]
## c1    3    4    5    6
## c2    5    6    7    8
## r3    9   10   11   12
\end{verbatim}

It's good to know that the following expressions result in NaNs and
infinite.

\begin{Shaded}
\begin{Highlighting}[]
\DecValTok{0}\SpecialCharTok{/}\DecValTok{0}
\end{Highlighting}
\end{Shaded}

\begin{verbatim}
## [1] NaN
\end{verbatim}

\begin{Shaded}
\begin{Highlighting}[]
\ConstantTok{Inf}\SpecialCharTok{/}\ConstantTok{Inf}
\end{Highlighting}
\end{Shaded}

\begin{verbatim}
## [1] NaN
\end{verbatim}

\begin{Shaded}
\begin{Highlighting}[]
\ConstantTok{Inf}\SpecialCharTok{/}\DecValTok{0}
\end{Highlighting}
\end{Shaded}

\begin{verbatim}
## [1] Inf
\end{verbatim}

\hypertarget{list}{%
\paragraph{List}\label{list}}

R lists are versatile containers since, unlike atomic vectors, they can
have mixed data types. Aka generic vectors, lists are (in general)
heterogeneous data containers which can also have another list as
element (nested list).

A list is a special type of vector. Each element can be a different
type.

\begin{itemize}
\tightlist
\item
  To create a list: list()
\item
  To convert objects to list: as.list().
\end{itemize}

An empty list of the required length can be created using vector()

\hypertarget{create-a-list}{%
\subparagraph{Create a list}\label{create-a-list}}

\begin{Shaded}
\begin{Highlighting}[]
\NormalTok{x }\OtherTok{\textless{}{-}} \FunctionTok{list}\NormalTok{(}\DecValTok{5}\NormalTok{, }\StringTok{"hola"}\NormalTok{, }\ConstantTok{TRUE}\NormalTok{, }\FloatTok{1.618}\NormalTok{)}
\NormalTok{x}
\end{Highlighting}
\end{Shaded}

\begin{verbatim}
## [[1]]
## [1] 5
## 
## [[2]]
## [1] "hola"
## 
## [[3]]
## [1] TRUE
## 
## [[4]]
## [1] 1.618
\end{verbatim}

\hypertarget{select-list-elements}{%
\subparagraph{Select list elements}\label{select-list-elements}}

A list's element can be referenced using double square brackets. NB
index starts from 1.

\begin{Shaded}
\begin{Highlighting}[]
\NormalTok{x[[}\DecValTok{1}\NormalTok{]]}
\end{Highlighting}
\end{Shaded}

\begin{verbatim}
## [1] 5
\end{verbatim}

Since a list is a special type of vector, it can be created also with
vector().

\begin{Shaded}
\begin{Highlighting}[]
\CommentTok{\#create an empty list}
\NormalTok{x }\OtherTok{\textless{}{-}} \FunctionTok{vector}\NormalTok{(}\StringTok{"list"}\NormalTok{, }\AttributeTok{length =} \DecValTok{4}\NormalTok{)}
\FunctionTok{length}\NormalTok{(x)}
\end{Highlighting}
\end{Shaded}

\begin{verbatim}
## [1] 4
\end{verbatim}

\begin{Shaded}
\begin{Highlighting}[]
\NormalTok{x[[}\DecValTok{1}\NormalTok{]]}
\end{Highlighting}
\end{Shaded}

\begin{verbatim}
## NULL
\end{verbatim}

Convert vector to list.

\begin{Shaded}
\begin{Highlighting}[]
\NormalTok{x }\OtherTok{\textless{}{-}} \DecValTok{1}\SpecialCharTok{:}\DecValTok{5}
\NormalTok{x }\OtherTok{\textless{}{-}} \FunctionTok{as.list}\NormalTok{(x)}
\FunctionTok{length}\NormalTok{(x)}
\end{Highlighting}
\end{Shaded}

\begin{verbatim}
## [1] 5
\end{verbatim}

\hypertarget{lists-elements-names}{%
\paragraph{Lists: elements names}\label{lists-elements-names}}

Elements of a list can be named (i.e.~lists can have the names
attribute)

\begin{Shaded}
\begin{Highlighting}[]
\NormalTok{l}\OtherTok{\textless{}{-}} \FunctionTok{list}\NormalTok{(}\AttributeTok{a =} \StringTok{"Ciao"}\NormalTok{, }\AttributeTok{b =} \DecValTok{1}\SpecialCharTok{:}\DecValTok{5}\NormalTok{, }\AttributeTok{data =} \FunctionTok{head}\NormalTok{(iris))}
\FunctionTok{names}\NormalTok{(l)}
\end{Highlighting}
\end{Shaded}

\begin{verbatim}
## [1] "a"    "b"    "data"
\end{verbatim}

\begin{Shaded}
\begin{Highlighting}[]
\NormalTok{l}\SpecialCharTok{$}\NormalTok{a}
\end{Highlighting}
\end{Shaded}

\begin{verbatim}
## [1] "Ciao"
\end{verbatim}

\begin{Shaded}
\begin{Highlighting}[]
\NormalTok{l[[}\StringTok{"b"}\NormalTok{]]}
\end{Highlighting}
\end{Shaded}

\begin{verbatim}
## [1] 1 2 3 4 5
\end{verbatim}

\begin{Shaded}
\begin{Highlighting}[]
\CommentTok{\#Using a single bracket another list is returned label+element}
\NormalTok{l[}\DecValTok{1}\NormalTok{]}
\end{Highlighting}
\end{Shaded}

\begin{verbatim}
## $a
## [1] "Ciao"
\end{verbatim}

Functions in R returns only one object, so lists can be handy since they
are versatile containers that can have labels for elements.

On the console each element of a printed list starts on a new line.

\hypertarget{dataframes}{%
\subsubsection{Dataframes}\label{dataframes}}

It is maybe the most popular data structure, perfectly suited for
tabular datasets. It is like a list with each element having the same
length, i.e.~the column. It is the heterogeneous counterpart of a matrix
since columns can be of different data type. Also within the same column
elements can be of different types.

\hypertarget{create-a-dataframe}{%
\subparagraph{Create a dataframe}\label{create-a-dataframe}}

\begin{Shaded}
\begin{Highlighting}[]
\NormalTok{df }\OtherTok{\textless{}{-}} \FunctionTok{data.frame}\NormalTok{(}\AttributeTok{id =}\NormalTok{ letters[}\DecValTok{1}\SpecialCharTok{:}\DecValTok{5}\NormalTok{], }\AttributeTok{A =} \DecValTok{1}\SpecialCharTok{:}\DecValTok{5}\NormalTok{, }\AttributeTok{B =} \DecValTok{6}\SpecialCharTok{:}\DecValTok{10}\NormalTok{)}
\NormalTok{df }\OtherTok{\textless{}{-}} \FunctionTok{data.frame}\NormalTok{(}\AttributeTok{id =}\NormalTok{ letters[}\DecValTok{1}\SpecialCharTok{:}\DecValTok{5}\NormalTok{], }\AttributeTok{A =} \FunctionTok{c}\NormalTok{(}\DecValTok{1}\NormalTok{,}\StringTok{\textquotesingle{}dos\textquotesingle{}}\NormalTok{,}\ConstantTok{TRUE}\NormalTok{,}\DecValTok{4}\NormalTok{,}\StringTok{\textquotesingle{}cinque\textquotesingle{}}\NormalTok{), }\AttributeTok{B =} \DecValTok{6}\SpecialCharTok{:}\DecValTok{10}\NormalTok{)}
\NormalTok{df}
\end{Highlighting}
\end{Shaded}

\begin{verbatim}
##   id      A  B
## 1  a      1  6
## 2  b    dos  7
## 3  c   TRUE  8
## 4  d      4  9
## 5  e cinque 10
\end{verbatim}

\begin{Shaded}
\begin{Highlighting}[]
\CommentTok{\#check if it is a list}
\FunctionTok{is.list}\NormalTok{(df)}
\end{Highlighting}
\end{Shaded}

\begin{verbatim}
## [1] TRUE
\end{verbatim}

\begin{Shaded}
\begin{Highlighting}[]
\FunctionTok{class}\NormalTok{(df)}
\end{Highlighting}
\end{Shaded}

\begin{verbatim}
## [1] "data.frame"
\end{verbatim}

\begin{Shaded}
\begin{Highlighting}[]
\FunctionTok{typeof}\NormalTok{(df)}
\end{Highlighting}
\end{Shaded}

\begin{verbatim}
## [1] "list"
\end{verbatim}

\hypertarget{useful-functions-with-dataframes}{%
\subparagraph{Useful functions with
dataframes}\label{useful-functions-with-dataframes}}

Some useful functions with dataframes follow.

\begin{Shaded}
\begin{Highlighting}[]
\CommentTok{\#shows first n rows}
\FunctionTok{head}\NormalTok{(df,}\AttributeTok{n=}\DecValTok{2}\NormalTok{) }
\end{Highlighting}
\end{Shaded}

\begin{verbatim}
##   id   A B
## 1  a   1 6
## 2  b dos 7
\end{verbatim}

\begin{Shaded}
\begin{Highlighting}[]
\CommentTok{\#shows last n rows}
\FunctionTok{tail}\NormalTok{(df,}\AttributeTok{n=}\DecValTok{3}\NormalTok{)}
\end{Highlighting}
\end{Shaded}

\begin{verbatim}
##   id      A  B
## 3  c   TRUE  8
## 4  d      4  9
## 5  e cinque 10
\end{verbatim}

\begin{Shaded}
\begin{Highlighting}[]
\CommentTok{\#returns the dimensions of data frame (i.e. number of rows and number of columns)}
\FunctionTok{dim}\NormalTok{(df) }
\end{Highlighting}
\end{Shaded}

\begin{verbatim}
## [1] 5 3
\end{verbatim}

\begin{Shaded}
\begin{Highlighting}[]
\CommentTok{\#number of rows}
\FunctionTok{nrow}\NormalTok{(df) }
\end{Highlighting}
\end{Shaded}

\begin{verbatim}
## [1] 5
\end{verbatim}

\begin{Shaded}
\begin{Highlighting}[]
\CommentTok{\#number of columns}
\FunctionTok{ncol}\NormalTok{(df) }
\end{Highlighting}
\end{Shaded}

\begin{verbatim}
## [1] 3
\end{verbatim}

\begin{Shaded}
\begin{Highlighting}[]
\CommentTok{\# structure of data frame: name, type and preview for each column}
\FunctionTok{str}\NormalTok{(df) }
\end{Highlighting}
\end{Shaded}

\begin{verbatim}
## 'data.frame':    5 obs. of  3 variables:
##  $ id: Factor w/ 5 levels "a","b","c","d",..: 1 2 3 4 5
##  $ A : Factor w/ 5 levels "1","4","cinque",..: 1 4 5 2 3
##  $ B : int  6 7 8 9 10
\end{verbatim}

\begin{Shaded}
\begin{Highlighting}[]
\CommentTok{\#names or colnames() to show the names attribute for a data frame}
\FunctionTok{names}\NormalTok{(df) }
\end{Highlighting}
\end{Shaded}

\begin{verbatim}
## [1] "id" "A"  "B"
\end{verbatim}

\begin{Shaded}
\begin{Highlighting}[]
\FunctionTok{colnames}\NormalTok{(df)}
\end{Highlighting}
\end{Shaded}

\begin{verbatim}
## [1] "id" "A"  "B"
\end{verbatim}

\begin{Shaded}
\begin{Highlighting}[]
\CommentTok{\#shows the class of each column in the data frame}
\FunctionTok{sapply}\NormalTok{(df, class) }
\end{Highlighting}
\end{Shaded}

\begin{verbatim}
##        id         A         B 
##  "factor"  "factor" "integer"
\end{verbatim}

\begin{Shaded}
\begin{Highlighting}[]
\CommentTok{\#the sapply() function takes list, vector or data frame as input and returns vector or matrix as output.}
\end{Highlighting}
\end{Shaded}

\hypertarget{selecting-elements-1}{%
\subparagraph{Selecting elements}\label{selecting-elements-1}}

Similarly to the matrix case, both a row and a column indentifiers must
be specified.

\begin{Shaded}
\begin{Highlighting}[]
\NormalTok{df[}\DecValTok{1}\NormalTok{, }\DecValTok{3}\NormalTok{]}
\end{Highlighting}
\end{Shaded}

\begin{verbatim}
## [1] 6
\end{verbatim}

Since data frames are also lists, we can use the list notation: double
square brackets or \$.

The columns are in fact elements of such list.

\begin{Shaded}
\begin{Highlighting}[]
\NormalTok{df[[}\StringTok{"A"}\NormalTok{]]}
\end{Highlighting}
\end{Shaded}

\begin{verbatim}
## [1] 1      dos    TRUE   4      cinque
## Levels: 1 4 cinque dos TRUE
\end{verbatim}

\begin{Shaded}
\begin{Highlighting}[]
\NormalTok{df}\SpecialCharTok{$}\NormalTok{A}
\end{Highlighting}
\end{Shaded}

\begin{verbatim}
## [1] 1      dos    TRUE   4      cinque
## Levels: 1 4 cinque dos TRUE
\end{verbatim}

\hypertarget{factors}{%
\subsubsection{Factors}\label{factors}}

A factor is a data structure specifically built for categorical
variables that take on only a predefined number of values eg color =
``yellow'', ``blue'', ``red''. It can also be ordered eg humidity =
`low', `medium', `high'. These 2 examples have both 3 levels (unique
categorical values).

\hypertarget{create-a-factor}{%
\paragraph{Create a factor}\label{create-a-factor}}

\begin{Shaded}
\begin{Highlighting}[]
\CommentTok{\#explicit level declaration}
\NormalTok{x }\OtherTok{\textless{}{-}} \FunctionTok{factor}\NormalTok{(}\FunctionTok{c}\NormalTok{(}\StringTok{"single"}\NormalTok{, }\StringTok{"married"}\NormalTok{, }\StringTok{"married"}\NormalTok{, }\StringTok{"married"}\NormalTok{, }\StringTok{"single"}\NormalTok{), }
            \AttributeTok{levels =} \FunctionTok{c}\NormalTok{(}\StringTok{"single"}\NormalTok{, }\StringTok{"married"}\NormalTok{, }\StringTok{"divorced"}\NormalTok{));}
\NormalTok{x}
\end{Highlighting}
\end{Shaded}

\begin{verbatim}
## [1] single  married married married single 
## Levels: single married divorced
\end{verbatim}

\begin{Shaded}
\begin{Highlighting}[]
\CommentTok{\#if levels are not provided R infers them from the data provided}
\NormalTok{x2 }\OtherTok{\textless{}{-}} \FunctionTok{factor}\NormalTok{(}\FunctionTok{c}\NormalTok{(}\StringTok{"blue"}\NormalTok{, }\StringTok{"yellow"}\NormalTok{, }\StringTok{"red"}\NormalTok{, }\StringTok{"yellow"}\NormalTok{, }\StringTok{"blue"}\NormalTok{))}
\NormalTok{x2}
\end{Highlighting}
\end{Shaded}

\begin{verbatim}
## [1] blue   yellow red    yellow blue  
## Levels: blue red yellow
\end{verbatim}

\begin{Shaded}
\begin{Highlighting}[]
\CommentTok{\#to add a level}
\FunctionTok{levels}\NormalTok{(x) }\OtherTok{\textless{}{-}} \FunctionTok{c}\NormalTok{(}\FunctionTok{levels}\NormalTok{(x), }\StringTok{"complicated"}\NormalTok{) }
\NormalTok{x}
\end{Highlighting}
\end{Shaded}

\begin{verbatim}
## [1] single  married married married single 
## Levels: single married divorced complicated
\end{verbatim}

\begin{Shaded}
\begin{Highlighting}[]
\NormalTok{x[}\DecValTok{2}\NormalTok{]}
\end{Highlighting}
\end{Shaded}

\begin{verbatim}
## [1] married
## Levels: single married divorced complicated
\end{verbatim}

\end{document}
